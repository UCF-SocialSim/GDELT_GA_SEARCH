%!TEX root =  GDELT_GA_Search_UsersManual.tex

\chapter{Spatial Tuning}
\label{chapter:SpatialTuning}

\textbf{Spatial Tuning} means customizing the way the search algorithm explores a particular dimension.
For example, two fields that are present in the GDELT event data are \textit{latitude} and \textit{longitude}.
When parsing the GDELT data, the gdelt\_ga\_search algorithm separates events by geographic location.
This is done with a simple mathematical approach: latitude and longitude are used to form a `box' on the 
Earth's surface. 
Events are categorized as either \textit{inside} or \textit{outside} that box. 
It is possible to expand and shrink the box for the purpose of growing or shrinking the number of events found \textit{inside} the search region.

\par Events in a data set are not evenly distributed on the Earth's surface. 
Of areas that have events, some contain a large or small number of events.
Small areas with many events are \textit{high-density} areas, while areas with few events are \textit{low-density areas.}
Of course, there are places without events. 
For example, there are virtually no events that occur in the middle of the Pacific Ocean. 
Many other areas do not have events, which can be seen in data sets for specific geographic regions. 

\par Exploring areas that are not represented in the data set can waste time. If there are no events in a region, the region can be disregarded.

\par If events are low-density in some areas and high-density in other areas, use a more nuanced strategy. 
For a box with mostly low-density areas, use large increases (i.e, expand the box by hundreds of miles) to include a few more events by expanding the size of the search box.
For a box with mostly high-density areas, use small increases (i.e., expand the box by a few miles). Small increases may need to be done multiple times. 
If moving the boundaries even a few miles means including or excluding thousands of events, then a better strategy is to use small icreases.

\par The above logic applies to other fields in the GDELT data. One of the other GDELT fields is /textit{Root Code}, which has values from 1 to 20. 
Just as events are subsetted by latitude/longitude, codes can also be used to subset the data set.
For an example with only one dimension (so instead of a 2-D box there is only a range), consider a search range with event codes 7 through 11.

\par To expand the search for more event types, the code range could be expanded by moving one of the boundaries. For example, the range could expand to 6-11 or  to 7-12, or by moving both of the boundaries, to 6-12.

\par The search can also be made smaller by restricting the code range. 
Using the above example, to search for fewer event types, the range could be decreased to 8-11, 7-10, 3-8, etc.

\par However, if it is known that there are no events labeled by a specific code, then that code should be skipped. 
Using the above example of the range 7-11, if there are no code `6' events, then there is no reason to search for '6'. Instead, it is better skip 6 and use 5-11 instead of 7-11.

\par Spatial tuning refers to the collection of strategies to optimize the way that the gdelt\_ga\_search algorithm expands and restricts its search ranges on specific domains, given what is known \textit{a priori} about the distribution of events on those domains in a data set.

