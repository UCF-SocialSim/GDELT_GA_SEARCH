 %!TEX root =  GDELT_GA_Search_UsersManual.tex

\chapter{Getting Started: Running \gdgas Searches} \label{chap:Running}

This chapter provides a basic introduction to running the simulation 	

\section{Input Data}
The input data for \gdgas searches consists of four kinds of files:

\begin{itemize}
\item GDELT Data, including:
\begin{itemize}
\item Events Data
\item Mentions Data (can be omitted)
\end{itemize}
\item Time Series Comparative Data
\item The GDELT Properties File
\item Run Group Specification File (for sets of parallelized runs)
\end{itemize}

The GDELT Properties File and the Run Group Specification File are used with single runs and with parallelized runs, which are both discussed in subsequent sections

\subsection{GDELT Data}
GDELT data consists of \textit{events} and \textit{mentions}. Both of these are given in CSV files downloadable from GDELT's archives. This toolkit provides a pair of tools that can automatically download events and related mentions from within a given time frame. (Note that mentions may accumulate after the initial event, and thus some events' mentions may fall outside the original time limit for events). An alternative data structure in JSON, from the original project for which this tool was designed, is also accommodated but is not documented.

\subsection{Time Series Comparative Data}
The search will compare the GDELT data with a time series; the input data for the time series has the following structure (for more information, see \ref{TimeSeriesDataStructure}).

\begin{verbatim}
platform,frame,ConvertedDate,counts
twitter,capitol,2020-10-01,1034
twitter,capitol,2020-10-02,1081
twitter,capitol,2020-10-03,1096
twitter,capitol,2020-10-04,1116
twitter,capitol,2020-10-05,1198
twitter,capitol,2020-10-06,1234
twitter,capitol,2020-10-07,1271
twitter,capitol,2020-10-08,1368
etc.
\end{verbatim}


\section{Running Single Runs}

\subsection{The GDELT properties file}\label{sec:GDELTPropertiesFile}

The GDELT properties file describes the location of the GDELT data, its chronological boundaries, the parameters of the GDELT queries (including their mutation properties), the location of the social media data, and some other properties of exploration. The structure of the file is simple: each line can contain one property specification, with a non-whitespaced property name on the left, an equals sign, and the property value on the right. Lines that start with number signs are ignored and can be used as comments.

Here is an example of a file

\begin{lstlisting}
# GIT DIRECTORY
gitDir = /home/username/gdelt_ga_search

# GDELT DATA FILE LOCATION AND INFO
dataLocation     = ./data/examples/GDELT_Events_Data.csv
mentionsLocation = ./data/examples/GDELT_Mentions_Data.csv
beginHistory     = 2020-01-31
dayZero          = 2021-01-01  

# GDELT QUERY SPECIFICATIONS
MaxShift              = 24
MinShift              = 24
MaxVizOffset          = 0
MinVizOffset          = 0
Actor1CountryCodes    = null,KEN,ZWE,UGA,AFR
Actor2CountryCodes    = null,KEN,ZWE,UGA,AFR
MatchGoldsteinScale   = FALSE
MatchAverageTone      = FALSE
MatchRootCode         = FALSE
MatchLatLon           = FALSE

# MUTATION WEIGHTS
Actor1CountryCodesMutateWeight = 10
Actor2CountryCodesMutateWeight = 10

# TIME SHIFT MUTATION PROBABILITY
MutatingTimeshiftProb           =  0

# TIME SERIES DATA FILE LOCATION AND INFO
timeSeriesDataLocation = ./examples/DummySocialMediaCounts.csv
timeSeriesDataStart    = 2020-10-01T00:00:00

# BAILOUT
bailoutInterval  = 3600
bailoutValue     = 100
\end{lstlisting}

\begin{itemize}
\item \textit{gitDir} This specifies the directory of the GDELT code. In cases where the code may be modified, it can be useful to employ a mechanism that will stop a run if the code has not been committed.
\item \textit{dataLocation} The location of the GDELT events data file.
\item \textit{mentionsLocation} The location of the GDELT mentions data file. This can be omitted if no mentions are available
\item \textit{beginHistory} The earliest date in the GDELT data
\item \textit{dayZero} The date that will be considered the boundary between training and prediction  
\item \textit{MaxShift} Maximum number of hours the GDELT search will `shift' times. This can be used so that predictions are not (or, occasionally, are) based on information from the future, e.g. a 24-hour shift means the predictions are based on the day prior.
\item \textit{MinShift} Minimum number of hours to shift. If this value is the same as the MaxShift, the shift will be constant
\item \textit{MaxVizOffset} Maximum visibility offset, or the number of hours offset the visibility of events will be
\item \textit{MinVizOffset} Minimum visibility offset
\item \textit{Actor1CountryCodes} List of country codes that are used for Actor 1 Country Code queries
\item \textit{Actor2CountryCodes} List of country codes that are used for Actor 2 Country Code queries
\item \textit{MatchGoldsteinScale} Whether to consider Goldstein Scale when matching queries
\item \textit{MatchAverageTone} Whether to consider Average Tone when matching queries
\item \textit{MatchRootCode} Whether to consider Root Code when matching queries
\item \textit{MatchLatLon} Whether to consider Latitude and Longitude when matching queries
\item \textit{Actor1CountryCodesMutateWeight} Weighted probability that the Actor 1 Country Code element of a query will mutate
\item \textit{Actor2CountryCodesMutateWeight} Weighted probability that the Actor 2 Country Code element of a query will mutate
\item \textit{MutatingTimeshiftProb} Probability that the shift value will mutate
\item \textit{timeSeriesDataLocation} File location for the social media counts
\item \textit{timeSeriesDataStart} Earliest date in the social media files.
\item \textit{bailoutInterval} Time (in seconds) after which the routine will stop a search if no improvement has been found
\item \textit{bailoutValue} Number of iterations after which the routine will stop a search is no improvement has been found
\end{itemize}

\subsection{Using the Runner Class from the Command Line}

It is possible to use the Runner class to run single runs of the search. This requires a collection of arguments. A typical command line will include arguments following this example:

\begin{itemize}
\item \textit{ /usr/lib/jvm/java-8-openjdk-amd64/bin/java}  is the path to the Java RunTime
\item \textit{ -DGDELT\_GAS\_PSCRIPT\_PATH=``/home/gdelt\_ga\_search/pscripts/"}  is the path to the Python Scripts. This is required even if Python Scripts are not used
\item \textit{-Dfile.encoding=UTF-8} A specification of file encoding
\item \textit{-classpath ./bin:./lib/gson-2.8.5.jar:./lib/jeromq-0.4.0.jar:./lib/commons-math3-3.6.1.jar gdelt.runners.Runner} The classpath
\item \textit{RUNNER\_0} An identifier that is used for some interim files
\item \textit{./examples/SampleConfig.props} The properties file in which the characteristics of the input data and queries are specified
\item \textit{LINEAR\_REGRESSION} The type of prediction to be used
\item \textit{NRMSE\_MEAN} The type of scoring to be used
\item \textit{2021-01-01} The date to be used as day zero
\item \textit{28} The duration of the training period
\item \textit{21} The number of days forward to predict for the final prediction
\item \textit{21} The number of days to predict for the training periods
\item \textit{50 }The population size
\item \textit{10} The number of survivors
\item \textit{15} The number of generations to run if no stopping condition is encountered
\item \textit{7} The number of days prior to day zero minus the test prediction period to take as the training period day zero
\item \textit{twitter\_capitol} the data source ('platform') and topic ('frame') to be used; must match the input data given in the configuration properties file named above 
\item \textit{DEMO\_001\_000000::NewQueries} An alphanumeric identifier and a description; the description can use underscores as spaces. The description will be written with every output file.
\end{itemize}

\section{Running Parallelized Suites of Runs}

\subsection{The Run Group Specification File}

The Run Group Specification File contains the information needed to structure a collection of parallelized runs. The format is a property on the left, a pair of colons, and a value on the right. The term on the right may include several different values. The configuration will create a set of runs based on all of the possible combinations of values.

An example listing of a Run Group Specification File (from ./examples/SampleConfig.props).
\begin{lstlisting}
predMethods::LINEAR_REGRESSION
scoreMethods::NRMSE_MEAN
zeroTimes::2021-01-01
trainDurations::28
predDurations::21
predDurationTrains::21
popSizes::50
survivorValues::10
generations::15
predictionPoints::7,0
platforms::twitter,youtube
frames::capitol,insurrection
\end{lstlisting}

The meanings of these are:

\begin{itemize}
\item \textit{predMethods} The method to be used for prediction; valid values are the names of the values in the PREDICTION\_METHOD enum of the Predictor class.
\item \textit{scoreMethods} The method of scoring to be used; value values are the names of the values in the SCORE\_METHOD enum of the Scorer class.
\item \textit{zeroTimes} Times that are to be used as the boundary between training and prediction
\item \textit{trainDurations} Number of days that are used for training periods
\item \textit{predDurations} Number of days forward (ahead of zero time) that is considered prediction
\item \textit{predDurationTrains} Number of days forward used by the training samples. 
\item \textit{popSizes} Size of the GA population
\item \textit{survivorValues} Number of best-scoring survivors in the population
\item \textit{generations} Number of generations the search will run if it does not hit a stopping condition first
\item \textit{predictionPoints} Days prior to zero time \textit{minus the predDurationTrains} value that will mark the train/test boundary for the predictions.
\item \textit{platforms} Social media platforms to be explored
\item \textit{frames} Frames (topics) to be explored
\end{itemize}

In the above example, eight combinations would be created based on the fact that multiple values for prediction points, platforms, and frames are combined.

\subsection{Running The ParallelMultiRunner Class from the Command Line}

When running the ParallelMultiRunner from the command line, the arguments should be
\begin{itemize}
\item The full path to your Java runtime, e.g. /usr/lib/jvm/java-8-openjdk-amd64/bin/java 
\item The relative path to your GDELT Properties file, e.g. ./examples/SampleConfig.props 
\item A group name for this set of runs, e.g. DEMO\_001 
\item A description for this group of runs, surrounded by quotation marks, e.g. ``Queries with  Actor 1 and Actor 2 fewer test periods, longer training periods, larger population, fewer survivors"
\item The number of repetitions to run, i.e. the number of duplicate runs for every combination in the Run Group Configuration File (e.g., 2)
\item The number of processes available for parallelization (e.g. 2)
\item The relative path to the Run Group Configuration file, e.g. examples/Example\_Spec.rgspec
\end{itemize}

